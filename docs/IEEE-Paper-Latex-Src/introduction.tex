%!TEX root=main.tex

\section{Introduction} \label{intro}

In recent years, the number and severity of cyber-attacks has grown significantly \cite{Symantec-Threat-Report, IBM-XForce-Report}, \cite{IBM-XForce-Report}. Attackers have managed to pull off massive virtual bank heists, distributed denial of service (DDoS) attacks powered by botnets of the Internet of Things (IoT) devices, and even malware-cause power outages \cite{IBM-XForce-Report}. The National Science Foundation (NSF) SecureCloud test bed aims to combat the growing number of cyber-attacks against cloud networks using an autonomic, zero trust environment \cite{7796146}.  Our most recent implementation utilizes new software developed for use in an Observe Orient Decide Act (OODA) control plane.

Part of this system uses the GStar graph database application to collect and analyze data. Specifically, GStar makes use of a PostGreSQL object relational database, which allows us to perform mathematical analysis on cyber-attack data \cite{GStar}.  After making our initial work available online, we observed a number of unauthorized connection attempts against the GStar Application Programming Interface (API).  More specifically, the attacks targeted GStar's REpresentational State Transfer (REST) API, which is among the most commonly used API architectures \cite{REST-API-use}. There are many examples of recent attacks against REST APIs, including well publicized attacks on the Nissan Leaf smart car \cite{Nissan-Leaf} and the Internal Revenue Service database \cite{IRS}.  Existing APIs often do not follow security best practices, and developers can be lulled into a false sense of security believing that the API will not be an attack target.  Given the recent unsolicited attacks against GStar, we felt it was appropriate to create a more structured profile of these attacks, in an effort to improve the security of our system and other REST APIs.

To accomplish this, we created a low-interaction API honeypot to collect data on these attacks.  Honeypots are servers or systems built and deployed to mimic critical parts of a network, effectively distracting attackers and logging attack information in the process \cite{honeypot-Def}. Disguising a honeypot as an API allows us to analyze and understand attack patterns.  Since our defensive response improves as we collect more data, we can effectively use the attackers' strengths against them; the more times our honeypot is attacked, the better our defense posture becomes.  Since our original REST API was not designed to comprehensively log attack data, we only captured the timestamp, source, command type, and command text for the initial attacks.  To address these shortcomings, we created a new API honeypot, aptly named Pasithea (the Greek goddess of rest), capable of creating an attack profile that includes the user agent, the IP address, and other the information extracted from GStar.

The information gathered from Pasithea can help us determine where attacks are coming from, how they can be classified, and what can be done to defend against such attacks.  Our API honeypot Pasithea has been deployed online, and data collected from this and other sources will be integrated into the NSF SecureCloud test bed project.  Our honeypot enables security experts to analyze and remedy emerging attacks, to avoid falling into framework-complacency.  While various types of honeypots have existed for some time \cite{Stoll:1989:CET:67554}, \cite{Provos:2004:VHF:1251375.1251376}, and many security projects use APIs to make their data more easily consumable \cite{Graham:2008:FAD:1355323},\cite{SCADA-Testbed-API}, the use of a REST API itself to attract malicious traffic and collect attack data does not appear to have been studied previously.  It is important to note a potential source of confusion regarding API honeypots. Honeypot APIs, though similar in name, are not to be confused with Pasithea, which is an API honeypot. A honeypot API is an API made to interface with a specific honeypot, returning useful data or executing certain methods specified by the request \cite{Honeypot-API}. An API honeypot such as Pasithea, on the other hand, is entirely different; it functions as a proper honeypot by gathering data on unauthorized API requests.  This type of API honeypot represents a novel approach to attack analysis that will hopefully lead to the development of more secure and robust APIs for cloud-based applications.

The remainder of this paper is organized as follows. Section II presents the analysis of the data received from both the GStar REST API logs and current data collected by Pasithea. Section III describes the software design and features of Pasithea. Lastly, Section IV presents the results from performance testing against Pasithea.

