%!TEX root=main.tex

\section{Conclusions and Future Work} \label{conclusions}

Today's API security landscape is more like a ``Wild West'' of conflicting standards than a safe, civilized, city of consistency.
This has led to an influx of attacks directed at APIs on many  fronts. 
Based on the attacks targeting G-star and our research into related attacks, we have constructed an API honeypot, Pasithea, with Java and NanoHTTPD to help combat and detail future attacks on the API landscape. 
Using hive plots and other graph and relational tools, we have analyzed a real-world attack on G-star, demonstrating how DDoS and XSS attacks can be uncovered and attributed so that an appropriate defense may be deployed.  
Our performance data suggests that Pasithea should be able to keep a malicious user interested with fast response times while also maintaining composure and stability under high traffic loads. 
This allows us to develop accurate API attack profiles which will help shape the future of API security.

Our next steps include extending Pasithea's REST interface to extract more data from attackers while still maintaining its cover as an unidentifiable API honeypot, reporting additional results as Pasithea spends more time in the ``Wild West'' of the Internet collecting more data, and exploring higher interaction versions of an API honeypot where we would be able to respond with artifical data.
